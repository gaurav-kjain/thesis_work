\subsection{Experiments}

In this section we analyze the effect of different initializers based on our experiment results.

\subsubsection{CIFAR100}
\textbf{Two Layer, opti=adagrad, batch size=1xc}

\pgfplotstableclear{\dnwonoptinit}
\pgfplotstableclear{\dnwtwoptinit}
\pgfplotstableclear{\dnwfooptinit}
\pgfplotstableclear{\dnwsxoptinit}
\pgfplotstableclear{\dnwegoptinit}
\pgfplotstableclear{\dnwhnoptinit}

\pgfplotstableread[col sep = comma]{ResultsNormal/cifar100_TwoLayer_100B_0E_nadam_gl_n.csv}\dnwonoptinit
\pgfplotstableread[col sep = comma]{ResultsNormal/cifar100_TwoLayer_100B_0E_nadam_gl_u.csv}\dnwtwoptinit
\pgfplotstableread[col sep = comma]{ResultsNormal/cifar100_TwoLayer_100B_0E_nadam_he_u.csv}\dnwfooptinit
\pgfplotstableread[col sep = comma]{ResultsNormal/cifar100_TwoLayer_100B_0E_nadam_norm.csv}\dnwsxoptinit
\pgfplotstableread[col sep = comma]{ResultsNormal/cifar100_TwoLayer_100B_0E_nadam_unif.csv}\dnwegoptinit
\pgfplotstableread[col sep = comma]{ResultsNormal/cifar100_TwoLayer_100B_0E_nadam_zero.csv}\dnwhnoptinit
\providecommand{\initerr}{Error for initial epochs:INIT}
\providecommand{\initacc}{Accuracy for initial epcohs}
\providecommand{\lateerr}{Error for later epochs}
\providecommand{\lateacc}{Accuracy for later epochs}
\providecommand{\boxerr}{Error variation}
\providecommand{\boxacc}{Accuracy variation}

\providecommand{\initfigtitle}{Different batch results for starting 15 epochs }
\providecommand{\initcaption}{to write}

\providecommand{\latefigtitle}{Different batch results for later epochs}
\providecommand{\latecaption}{to write}

\providecommand{\boxerraccfigtitle}{accuracy and error plot for full training epochs}
\providecommand{\boxerracccaption}{to write}

\providecommand{\trtscaption}{to write}
\providecommand	{\ymaxlim}{0.65}

\begin{figure}[H]
	\hspace{-1cm}
	\begin{tabular}{C{.5\textwidth}C{.5\textwidth}}%
		\subfigure [\initerr] {
			\begin{tikzpicture} %
			\begin{axis}[smooth,
			%xlabel={$epochs$}, %
			ylabel={$error$}, %
			height=4cm,%
			width=4cm,%
			legend style={font=\tiny,at={(1,1)},anchor=north east,fill=none, draw=none},%
			xmax=15,%
			yticklabel style={/pgf/number format/.cd,fixed,precision=5},%
			label style = {scale=0.8},%
			tick label style = {scale=0.8},%
			] %
			\addplot+[mark=pentagon*] table[x expr=\coordindex , y={val_loss}, col sep=comma] {\dnwonoptinit};
			\addlegendentry{1xc};%	
			\addplot+[mark=ball] table[x expr=\coordindex , y={val_loss}, col sep=comma] {\dnwtwoptinit};
			\addlegendentry{2xc};%	
			\addplot+[mark=10-pointed star] table[x expr=\coordindex , y={val_loss}, col sep=comma] {\dnwfooptinit};
			\addlegendentry{4xc};%							
			\addplot+[mark=triangle*] table[x expr=\coordindex , y={val_loss}, col sep=comma] {\dnwsxoptinit};
			\addlegendentry{6xc};%							
			\addplot+[mark=halfcircle*] table[x expr=\coordindex , y={val_loss}, col sep=comma] {\dnwegoptinit};
			\addlegendentry{8xc};%							
			\addplot+[mark=otimes*] table[x expr=\coordindex , y={val_loss}, col sep=comma] {\dnwhnoptinit};
			\addlegendentry{10xc};%
			\end{axis} %
			\end{tikzpicture}
		}&
			\subfigure [\initacc] {
			\begin{tikzpicture} %
			\begin{axis}[smooth,
			%xlabel={$epochs$}, %
			ylabel={$accuracy$}, %
			height=4cm,%
			width=4cm,%
			legend style={font=\tiny,at={(1,0)},anchor=south east,fill=none, draw=none},%
			xmax=15,%
			yticklabel style={/pgf/number format/.cd,fixed,precision=5},%
			label style = {scale=0.8},%
			tick label style = {scale=0.8},%
			] %
			\addplot+[mark=pentagon*] table[x expr=\coordindex , y={val_acc}, col sep=comma] {\dnwonoptinit};
			\addlegendentry{1xc};%	
			\addplot+[mark=ball] table[x expr=\coordindex , y={val_acc}, col sep=comma] {\dnwtwoptinit};
			\addlegendentry{2xc};%	
			\addplot+[mark=10-pointed star] table[x expr=\coordindex , y={val_acc}, col sep=comma] {\dnwfooptinit};
			\addlegendentry{4xc};%							
			\addplot+[mark=triangle*] table[x expr=\coordindex , y={val_acc}, col sep=comma] {\dnwsxoptinit};
			\addlegendentry{6xc};%							
			\addplot+[mark=halfcircle*] table[x expr=\coordindex , y={val_acc}, col sep=comma] {\dnwegoptinit};
			\addlegendentry{8xc};%							
			\addplot+[mark=otimes*] table[x expr=\coordindex , y={val_acc}, col sep=comma] {\dnwhnoptinit};
			\addlegendentry{10xc};%
			\end{axis} %
			\end{tikzpicture}
		}\\%
	\end{tabular}%
	\caption {\textit{\initfigtitle}}%
	\medskip
	\small
	\textit{\initcaption}
\end{figure}

\begin{figure}[H]
	\hspace{-1cm}
	\begin{tabular}{C{.5\textwidth}C{.5\textwidth}}%
		\subfigure [\boxerr] {
			\begin{tikzpicture} %
			\begin{axis}[%
			xlabel={\textit{batch size [in multiple of classes]}}, %
			ylabel={\textit{error}}, %
			boxplot/draw direction=y,%
			xtick={1,2,3,4,5,6},%
			xticklabels={1,2, 4, 6, 8, 10},%
			height=4cm,%
			width=6cm,%
			] %
			\addplot+ [boxplot,fill=cyan!10!white] table [x expr=1,y ={val_loss}] {\dnwonoptinit};							
			\addplot+ [boxplot,fill=green!20!white] table [x expr=2,y ={val_loss}] {\dnwtwoptinit};
			\addplot+ [boxplot,fill=orange!20!white] table [x expr=4,y ={val_loss}] {\dnwfooptinit};		
			\addplot+ [boxplot,fill=violet!10!white] table [x expr=6,y ={val_loss}] {\dnwsxoptinit};					
			\addplot+ [boxplot,fill=pink!10!white] table [x expr=8,y ={val_loss}]  {\dnwegoptinit};						
			\addplot+ [boxplot,fill=green!10!white] table [x expr=10,y ={val_loss}] {\dnwhnoptinit};
			\end{axis} %
			\end{tikzpicture}
		}&
		\subfigure [\boxacc] {
			\begin{tikzpicture} %
			\begin{axis}[%
			boxplot/draw direction=y,%
			xtick={1,2,3,4,5,6},%
			xticklabels={1,2, 4, 6, 8, 10},%
			xlabel={\textit{batch size [in multiple of classes]}}, %
			ylabel={\textit{accuracy}}, %
			height=4cm,%
			width=6cm,%
			] %
			\addplot+ [boxplot,fill=cyan!10!white] table [x expr=1,y ={val_acc}] {\dnwonoptinit};							
			\addplot+ [boxplot,fill=green!20!white] table [x expr=2,y ={val_acc}] {\dnwtwoptinit};
			\addplot+ [boxplot,fill=orange!20!white] table [x expr=4,y ={val_acc}] {\dnwfooptinit};		
			\addplot+ [boxplot,fill=violet!10!white] table [x expr=6,y ={val_acc}] {\dnwsxoptinit};					
			\addplot+ [boxplot,fill=pink!10!white] table [x expr=8,y ={val_acc}]  {\dnwegoptinit};						
			\addplot+ [boxplot,fill=green!10!white] table [x expr=10,y ={val_acc}] {\dnwhnoptinit};
			\end{axis}%
			\end{tikzpicture}
		}
	\end{tabular}
	\caption {\textit{\boxerraccfigtitle}}
	\medskip
	\small
	\textit{\boxerracccaption}
\end{figure}

\begin{figure}[H]
\hspace{-1cm}
\begin{tabular}{C{\textwidth}}%
	\subfigure [\lateerr] {
		\begin{tikzpicture} %
		\begin{axis}[smooth,
		xlabel={$epochs$}, %
		ylabel={$error$}, %
		height=3cm,%
		width=12cm,%
		legend style={font=\tiny,legend columns=-1, at={(1,1)},anchor=north east,fill=none, draw=none},%
		xmin=15,%
		yticklabel style={/pgf/number format/.cd,fixed,precision=5},%
		] %
		\addplot+[mark=pentagon*] table[x expr=\coordindex , y={val_loss}, col sep=comma] {\dnwonoptinit};
		\addlegendentry{1xc};%	
		\addplot+[mark=ball] table[x expr=\coordindex , y={val_loss}, col sep=comma] {\dnwtwoptinit};
		\addlegendentry{2xc};%	
		\addplot+[mark=10-pointed star] table[x expr=\coordindex , y={val_loss}, col sep=comma] {\dnwfooptinit};
		\addlegendentry{4xc};%							
		\addplot+[mark=triangle*] table[x expr=\coordindex , y={val_loss}, col sep=comma] {\dnwsxoptinit};
		\addlegendentry{6xc};%							
		\addplot+[mark=halfcircle*] table[x expr=\coordindex , y={val_loss}, col sep=comma] {\dnwegoptinit};
		\addlegendentry{8xc};%							
		\addplot+[mark=otimes*] table[x expr=\coordindex , y={val_loss}, col sep=comma] {\dnwhnoptinit};
		\addlegendentry{10xc};%
		\end{axis} %
		\end{tikzpicture}
	}\\
	\subfigure [\lateacc] {
		\begin{tikzpicture} %
		\begin{axis}[smooth,
		xlabel={$epochs$}, %
		ylabel={$accuracy$}, %
		height=3cm,%
		width=12cm,%
		legend style={font=\tiny,legend columns=-1, at={(1,0)},anchor=south east,fill=none, draw=none},%
		%ymin=\ymaxlim,%
		xmin=12,%
		yticklabel style={/pgf/number format/.cd,fixed,precision=5},%
		] %
		\addplot+[mark=pentagon*] table[x expr=\coordindex , y={val_acc}, col sep=comma] {\dnwonoptinit};
		\addlegendentry{1xc};%	
		\addplot+[mark=ball] table[x expr=\coordindex , y={val_acc}, col sep=comma] {\dnwtwoptinit};
		\addlegendentry{2xc};%	
		\addplot+[mark=10-pointed star] table[x expr=\coordindex , y={val_acc}, col sep=comma] {\dnwfooptinit};
		\addlegendentry{4xc};%							
		\addplot+[mark=triangle*] table[x expr=\coordindex , y={val_acc}, col sep=comma] {\dnwsxoptinit};
		\addlegendentry{6xc};%							
		\addplot+[mark=halfcircle*] table[x expr=\coordindex , y={val_acc}, col sep=comma] {\dnwegoptinit};
		\addlegendentry{8xc};%							
		\addplot+[mark=otimes*] table[x expr=\coordindex , y={val_acc}, col sep=comma] {\dnwhnoptinit};
		\addlegendentry{10xc};%
		\end{axis} %
		\end{tikzpicture}
	}\\
\end{tabular}
\caption {\textit{\latefigtitle}}
\medskip
\small
\textit{\latecaption}
\end{figure}

%

%%%%%%%%%%%%%%%%%%%%%%%%%%%%%%%%%%%%%%%%%%%%%%%%%%%%%%%START%%%%%%%%%%%%%%%%%%%%%%%%%%%%%%%%%%%%%%%%%%%%%%%%%%%
%%%%%%%%%%%%%%%%%%%%%%%%%%%%%%TRAINING VS TESTING ACCURACY AND LOSS FOR ALL BATCH SIZES%%%%%%%%%%%%%%%%%%%%%%%%%
\begin{figure}[H]
	\hspace{-1cm}
	\begin{tabular}{C{.50\textwidth}C{.50\textwidth}}%
		\subfigure [2x-Error] {
			\begin{tikzpicture} %
			\begin{axis}[smooth,
			label style = {scale=0.5},%
			tick label style = {scale=0.5},%
			xlabel={$training$}, %
			ylabel={$testing$}, %
			height=3cm,%
			width=6cm,%
			line width=1,%
			axis y line*=left,%
			legend style={legend columns=-1, at={(1,0)},anchor=south east,fill=red!10!white, draw=none},%
			] %
			\addplot+[blue!80!white,densely dotted,mark=1,fill=red!10!white] table[x={acc} , y={val_acc}] {\dnwtwoptinit};%
			\addlegendentry{accuracy};%						
			\end{axis} %
			\begin{axis}[smooth,
			label style = {scale=0.5},%
			tick label style = {scale=0.5},%
			%xlabel={$training-error$}, %
			%ylabel={$testing-error$}, %
			xticklabel shift={0.5cm},%
			axis line shift=3pt,%
			height=3cm,%
			width=6cm,%
			line width=1,%
			axis y line*=right,%
			legend style={legend columns=-1, at={(0,1)},anchor=north west,fill=blue!10!white, draw=none},%
			] %
			\addplot+[red!80!white,densely dotted,mark=1,fill=blue!10!white] table[x={loss} , y={val_loss}]{\dnwtwoptinit};%
			\addlegendentry{loss};%							
			\end{axis} %
			\end{tikzpicture}			
		}&
		\subfigure [2x-Accuracy] {
			\begin{tikzpicture} %
			\begin{axis}[smooth,
			label style = {scale=0.8},%
			xlabel={$training-error$}, %
			ylabel={$testing-error$}, %
			height=3cm,%
			width=6cm,%
			line width=1,%
			axis y line*=right,%
			legend style={legend columns=-1, at={(0,1)},anchor=north west,fill=blue!10!white, draw=none},%
			] %
			\addplot+[red!80!white,densely dotted,mark=1,fill=blue!10!white] table[x={loss} , y={val_loss}]{\dnwtwoptinit};%
			\addlegendentry{loss};%							
			\end{axis} %
			\end{tikzpicture}
		}\\%
	\subfigure [4x-Error] {
		\begin{tikzpicture} %
		\begin{axis}[smooth,
		xlabel={$training$}, %
		ylabel={$testing$}, %
		height=3cm,%
		width=6cm,%
	    line width=1,%
		] %
		\addplot+[blue!70!white,densely dotted,mark=square,fill=red!20!white] table[x={acc} , y={val_acc}] {\dnwfooptinit};				
		\end{axis} %
		\end{tikzpicture}%			
	}&
	\subfigure [4x-Accuracy] {
		\begin{tikzpicture} %
		\begin{axis}[smooth,
		xlabel={$training$}, %
		ylabel={$testing$}, %
		height=3cm,%
		width=6cm,%
		line width=1,%
		] %
		\addplot+[red!70!white,densely dotted,mark=square,fill=blue!20!white] table[x={loss} , y={val_loss}] {\dnwfooptinit};					
		\end{axis} %
		\end{tikzpicture}
	}\\%
	\subfigure [6x-Error] {
		\begin{tikzpicture} %
		\begin{axis}[smooth,
		xlabel={$training$}, %
		ylabel={$testing$}, %
		height=3cm,%
		width=6cm,%
		line width=1,%
		%ymin=0.992,%
		] %
		\addplot+[mark=square,fill=red!20!white] table[x={acc} , y={val_acc}] {\dnwsxoptinit};				
		\end{axis} %
		\end{tikzpicture}%			
	}&
	\subfigure [6x-Accuracy] {
		\begin{tikzpicture} %
		\begin{axis}[smooth,
		xlabel={$training$}, %
		ylabel={$testing$}, %
		height=3cm,%
		width=6cm,%
		line width=1,%
		] %
		\addplot+[mark=square,fill=red!20!white] table[x={loss} , y={val_loss}] {\dnwsxoptinit};					
		\end{axis} %
		\end{tikzpicture}
	}
	\end{tabular}
	\caption {\textit{\trtscaption}}
\end{figure}

%%%%%%%%%%%%%%%%%%%%%%%%%%%%%%TRAINING VS TESTING ACCURACY AND LOSS FOR ALL BATCH SIZES%%%%%%%%%%%%%%%%%%%%%%%%%
%%%%%%%%%%%%%%%%%%%%%%%%%%%%%%%%%%%%%%%%%%%%%%%%%%%%%%%END%%%%%%%%%%%%%%%%%%%%%%%%%%%%%%%%%%%%%%%%%%%%%%%%%%%


\pgfplotstableclear{\dnwonoptinit}
\pgfplotstableclear{\dnwtwoptinit}
\pgfplotstableclear{\dnwfooptinit}
\pgfplotstableclear{\dnwsxoptinit}
\pgfplotstableclear{\dnwegoptinit}
\pgfplotstableclear{\dnwhnoptinit}

\relax


\textbf{Two Layer, opti=SGD with nesterov momentum, batch size=1xc}

\pgfplotstableread[col sep = comma]{ResultsNormal/cifar100_TwoLayer_100B_0E_SGD_gl_n.csv}\dnwonoptinit
\pgfplotstableread[col sep = comma]{ResultsNormal/cifar100_TwoLayer_100B_0E_SGD_gl_u.csv}\dnwtwoptinit
\pgfplotstableread[col sep = comma]{ResultsNormal/cifar100_TwoLayer_100B_0E_SGD_he_u.csv}\dnwfooptinit
\pgfplotstableread[col sep = comma]{ResultsNormal/cifar100_TwoLayer_100B_0E_SGD_norm.csv}\dnwsxoptinit
\pgfplotstableread[col sep = comma]{ResultsNormal/cifar100_TwoLayer_100B_0E_SGD_unif.csv}\dnwegoptinit
\pgfplotstableread[col sep = comma]{ResultsNormal/cifar100_TwoLayer_100B_0E_SGD_zero.csv}\dnwhnoptinit
\renewcommand{\initerr}{Error for initial epochs:INIT}
\renewcommand{\initacc}{Accuracy for initial epcohs}
\renewcommand{\lateerr}{Error for later epochs}
\renewcommand{\lateacc}{Accuracy for later epochs}
\renewcommand{\boxerr}{Error variation}
\renewcommand{\boxacc}{Accuracy variation}

\renewcommand{\initfigtitle}{Different batch results for starting 15 epochs }
\renewcommand{\initcaption}{to write}

\renewcommand{\latefigtitle}{Different batch results for later epochs}
\renewcommand{\latecaption}{to write}

\renewcommand{\boxerraccfigtitle}{accuracy and error plot for full training epochs}
\renewcommand{\boxerracccaption}{to write}

\renewcommand{\trtscaption}{to write}
\renewcommand	{\ymaxlim}{0.65}

\begin{figure}[H]
	\hspace{-1cm}
	\begin{tabular}{C{.5\textwidth}C{.5\textwidth}}%
		\subfigure [\initerr] {
			\begin{tikzpicture} %
			\begin{axis}[smooth,
			%xlabel={$epochs$}, %
			ylabel={$error$}, %
			height=4cm,%
			width=4cm,%
			legend style={font=\tiny,at={(1,1)},anchor=north east,fill=none, draw=none},%
			xmax=15,%
			yticklabel style={/pgf/number format/.cd,fixed,precision=5},%
			label style = {scale=0.8},%
			tick label style = {scale=0.8},%
			] %
			\addplot+[mark=pentagon*] table[x expr=\coordindex , y={val_loss}, col sep=comma] {\dnwonoptinit};
			\addlegendentry{1xc};%	
			\addplot+[mark=ball] table[x expr=\coordindex , y={val_loss}, col sep=comma] {\dnwtwoptinit};
			\addlegendentry{2xc};%	
			\addplot+[mark=10-pointed star] table[x expr=\coordindex , y={val_loss}, col sep=comma] {\dnwfooptinit};
			\addlegendentry{4xc};%							
			\addplot+[mark=triangle*] table[x expr=\coordindex , y={val_loss}, col sep=comma] {\dnwsxoptinit};
			\addlegendentry{6xc};%							
			\addplot+[mark=halfcircle*] table[x expr=\coordindex , y={val_loss}, col sep=comma] {\dnwegoptinit};
			\addlegendentry{8xc};%							
			\addplot+[mark=otimes*] table[x expr=\coordindex , y={val_loss}, col sep=comma] {\dnwhnoptinit};
			\addlegendentry{10xc};%
			\end{axis} %
			\end{tikzpicture}
		}&
			\subfigure [\initacc] {
			\begin{tikzpicture} %
			\begin{axis}[smooth,
			%xlabel={$epochs$}, %
			ylabel={$accuracy$}, %
			height=4cm,%
			width=4cm,%
			legend style={font=\tiny,at={(1,0)},anchor=south east,fill=none, draw=none},%
			xmax=15,%
			yticklabel style={/pgf/number format/.cd,fixed,precision=5},%
			label style = {scale=0.8},%
			tick label style = {scale=0.8},%
			] %
			\addplot+[mark=pentagon*] table[x expr=\coordindex , y={val_acc}, col sep=comma] {\dnwonoptinit};
			\addlegendentry{1xc};%	
			\addplot+[mark=ball] table[x expr=\coordindex , y={val_acc}, col sep=comma] {\dnwtwoptinit};
			\addlegendentry{2xc};%	
			\addplot+[mark=10-pointed star] table[x expr=\coordindex , y={val_acc}, col sep=comma] {\dnwfooptinit};
			\addlegendentry{4xc};%							
			\addplot+[mark=triangle*] table[x expr=\coordindex , y={val_acc}, col sep=comma] {\dnwsxoptinit};
			\addlegendentry{6xc};%							
			\addplot+[mark=halfcircle*] table[x expr=\coordindex , y={val_acc}, col sep=comma] {\dnwegoptinit};
			\addlegendentry{8xc};%							
			\addplot+[mark=otimes*] table[x expr=\coordindex , y={val_acc}, col sep=comma] {\dnwhnoptinit};
			\addlegendentry{10xc};%
			\end{axis} %
			\end{tikzpicture}
		}\\%
	\end{tabular}%
	\caption {\textit{\initfigtitle}}%
	\medskip
	\small
	\textit{\initcaption}
\end{figure}

\begin{figure}[H]
	\hspace{-1cm}
	\begin{tabular}{C{.5\textwidth}C{.5\textwidth}}%
		\subfigure [\boxerr] {
			\begin{tikzpicture} %
			\begin{axis}[%
			xlabel={\textit{batch size [in multiple of classes]}}, %
			ylabel={\textit{error}}, %
			boxplot/draw direction=y,%
			xtick={1,2,3,4,5,6},%
			xticklabels={1,2, 4, 6, 8, 10},%
			height=4cm,%
			width=6cm,%
			] %
			\addplot+ [boxplot,fill=cyan!10!white] table [x expr=1,y ={val_loss}] {\dnwonoptinit};							
			\addplot+ [boxplot,fill=green!20!white] table [x expr=2,y ={val_loss}] {\dnwtwoptinit};
			\addplot+ [boxplot,fill=orange!20!white] table [x expr=4,y ={val_loss}] {\dnwfooptinit};		
			\addplot+ [boxplot,fill=violet!10!white] table [x expr=6,y ={val_loss}] {\dnwsxoptinit};					
			\addplot+ [boxplot,fill=pink!10!white] table [x expr=8,y ={val_loss}]  {\dnwegoptinit};						
			\addplot+ [boxplot,fill=green!10!white] table [x expr=10,y ={val_loss}] {\dnwhnoptinit};
			\end{axis} %
			\end{tikzpicture}
		}&
		\subfigure [\boxacc] {
			\begin{tikzpicture} %
			\begin{axis}[%
			boxplot/draw direction=y,%
			xtick={1,2,3,4,5,6},%
			xticklabels={1,2, 4, 6, 8, 10},%
			xlabel={\textit{batch size [in multiple of classes]}}, %
			ylabel={\textit{accuracy}}, %
			height=4cm,%
			width=6cm,%
			] %
			\addplot+ [boxplot,fill=cyan!10!white] table [x expr=1,y ={val_acc}] {\dnwonoptinit};							
			\addplot+ [boxplot,fill=green!20!white] table [x expr=2,y ={val_acc}] {\dnwtwoptinit};
			\addplot+ [boxplot,fill=orange!20!white] table [x expr=4,y ={val_acc}] {\dnwfooptinit};		
			\addplot+ [boxplot,fill=violet!10!white] table [x expr=6,y ={val_acc}] {\dnwsxoptinit};					
			\addplot+ [boxplot,fill=pink!10!white] table [x expr=8,y ={val_acc}]  {\dnwegoptinit};						
			\addplot+ [boxplot,fill=green!10!white] table [x expr=10,y ={val_acc}] {\dnwhnoptinit};
			\end{axis}%
			\end{tikzpicture}
		}
	\end{tabular}
	\caption {\textit{\boxerraccfigtitle}}
	\medskip
	\small
	\textit{\boxerracccaption}
\end{figure}

\begin{figure}[H]
\hspace{-1cm}
\begin{tabular}{C{\textwidth}}%
	\subfigure [\lateerr] {
		\begin{tikzpicture} %
		\begin{axis}[smooth,
		xlabel={$epochs$}, %
		ylabel={$error$}, %
		height=3cm,%
		width=12cm,%
		legend style={font=\tiny,legend columns=-1, at={(1,1)},anchor=north east,fill=none, draw=none},%
		xmin=15,%
		yticklabel style={/pgf/number format/.cd,fixed,precision=5},%
		] %
		\addplot+[mark=pentagon*] table[x expr=\coordindex , y={val_loss}, col sep=comma] {\dnwonoptinit};
		\addlegendentry{1xc};%	
		\addplot+[mark=ball] table[x expr=\coordindex , y={val_loss}, col sep=comma] {\dnwtwoptinit};
		\addlegendentry{2xc};%	
		\addplot+[mark=10-pointed star] table[x expr=\coordindex , y={val_loss}, col sep=comma] {\dnwfooptinit};
		\addlegendentry{4xc};%							
		\addplot+[mark=triangle*] table[x expr=\coordindex , y={val_loss}, col sep=comma] {\dnwsxoptinit};
		\addlegendentry{6xc};%							
		\addplot+[mark=halfcircle*] table[x expr=\coordindex , y={val_loss}, col sep=comma] {\dnwegoptinit};
		\addlegendentry{8xc};%							
		\addplot+[mark=otimes*] table[x expr=\coordindex , y={val_loss}, col sep=comma] {\dnwhnoptinit};
		\addlegendentry{10xc};%
		\end{axis} %
		\end{tikzpicture}
	}\\
	\subfigure [\lateacc] {
		\begin{tikzpicture} %
		\begin{axis}[smooth,
		xlabel={$epochs$}, %
		ylabel={$accuracy$}, %
		height=3cm,%
		width=12cm,%
		legend style={font=\tiny,legend columns=-1, at={(1,0)},anchor=south east,fill=none, draw=none},%
		%ymin=\ymaxlim,%
		xmin=12,%
		yticklabel style={/pgf/number format/.cd,fixed,precision=5},%
		] %
		\addplot+[mark=pentagon*] table[x expr=\coordindex , y={val_acc}, col sep=comma] {\dnwonoptinit};
		\addlegendentry{1xc};%	
		\addplot+[mark=ball] table[x expr=\coordindex , y={val_acc}, col sep=comma] {\dnwtwoptinit};
		\addlegendentry{2xc};%	
		\addplot+[mark=10-pointed star] table[x expr=\coordindex , y={val_acc}, col sep=comma] {\dnwfooptinit};
		\addlegendentry{4xc};%							
		\addplot+[mark=triangle*] table[x expr=\coordindex , y={val_acc}, col sep=comma] {\dnwsxoptinit};
		\addlegendentry{6xc};%							
		\addplot+[mark=halfcircle*] table[x expr=\coordindex , y={val_acc}, col sep=comma] {\dnwegoptinit};
		\addlegendentry{8xc};%							
		\addplot+[mark=otimes*] table[x expr=\coordindex , y={val_acc}, col sep=comma] {\dnwhnoptinit};
		\addlegendentry{10xc};%
		\end{axis} %
		\end{tikzpicture}
	}\\
\end{tabular}
\caption {\textit{\latefigtitle}}
\medskip
\small
\textit{\latecaption}
\end{figure}

%

%%%%%%%%%%%%%%%%%%%%%%%%%%%%%%%%%%%%%%%%%%%%%%%%%%%%%%%START%%%%%%%%%%%%%%%%%%%%%%%%%%%%%%%%%%%%%%%%%%%%%%%%%%%
%%%%%%%%%%%%%%%%%%%%%%%%%%%%%%TRAINING VS TESTING ACCURACY AND LOSS FOR ALL BATCH SIZES%%%%%%%%%%%%%%%%%%%%%%%%%
\begin{figure}[H]
	\hspace{-1cm}
	\begin{tabular}{C{.50\textwidth}C{.50\textwidth}}%
		\subfigure [2x-Error] {
			\begin{tikzpicture} %
			\begin{axis}[smooth,
			label style = {scale=0.5},%
			tick label style = {scale=0.5},%
			xlabel={$training$}, %
			ylabel={$testing$}, %
			height=3cm,%
			width=6cm,%
			line width=1,%
			axis y line*=left,%
			legend style={legend columns=-1, at={(1,0)},anchor=south east,fill=red!10!white, draw=none},%
			] %
			\addplot+[blue!80!white,densely dotted,mark=1,fill=red!10!white] table[x={acc} , y={val_acc}] {\dnwtwoptinit};%
			\addlegendentry{accuracy};%						
			\end{axis} %
			\begin{axis}[smooth,
			label style = {scale=0.5},%
			tick label style = {scale=0.5},%
			%xlabel={$training-error$}, %
			%ylabel={$testing-error$}, %
			xticklabel shift={0.5cm},%
			axis line shift=3pt,%
			height=3cm,%
			width=6cm,%
			line width=1,%
			axis y line*=right,%
			legend style={legend columns=-1, at={(0,1)},anchor=north west,fill=blue!10!white, draw=none},%
			] %
			\addplot+[red!80!white,densely dotted,mark=1,fill=blue!10!white] table[x={loss} , y={val_loss}]{\dnwtwoptinit};%
			\addlegendentry{loss};%							
			\end{axis} %
			\end{tikzpicture}			
		}&
		\subfigure [2x-Accuracy] {
			\begin{tikzpicture} %
			\begin{axis}[smooth,
			label style = {scale=0.8},%
			xlabel={$training-error$}, %
			ylabel={$testing-error$}, %
			height=3cm,%
			width=6cm,%
			line width=1,%
			axis y line*=right,%
			legend style={legend columns=-1, at={(0,1)},anchor=north west,fill=blue!10!white, draw=none},%
			] %
			\addplot+[red!80!white,densely dotted,mark=1,fill=blue!10!white] table[x={loss} , y={val_loss}]{\dnwtwoptinit};%
			\addlegendentry{loss};%							
			\end{axis} %
			\end{tikzpicture}
		}\\%
	\subfigure [4x-Error] {
		\begin{tikzpicture} %
		\begin{axis}[smooth,
		xlabel={$training$}, %
		ylabel={$testing$}, %
		height=3cm,%
		width=6cm,%
	    line width=1,%
		] %
		\addplot+[blue!70!white,densely dotted,mark=square,fill=red!20!white] table[x={acc} , y={val_acc}] {\dnwfooptinit};				
		\end{axis} %
		\end{tikzpicture}%			
	}&
	\subfigure [4x-Accuracy] {
		\begin{tikzpicture} %
		\begin{axis}[smooth,
		xlabel={$training$}, %
		ylabel={$testing$}, %
		height=3cm,%
		width=6cm,%
		line width=1,%
		] %
		\addplot+[red!70!white,densely dotted,mark=square,fill=blue!20!white] table[x={loss} , y={val_loss}] {\dnwfooptinit};					
		\end{axis} %
		\end{tikzpicture}
	}\\%
	\subfigure [6x-Error] {
		\begin{tikzpicture} %
		\begin{axis}[smooth,
		xlabel={$training$}, %
		ylabel={$testing$}, %
		height=3cm,%
		width=6cm,%
		line width=1,%
		%ymin=0.992,%
		] %
		\addplot+[mark=square,fill=red!20!white] table[x={acc} , y={val_acc}] {\dnwsxoptinit};				
		\end{axis} %
		\end{tikzpicture}%			
	}&
	\subfigure [6x-Accuracy] {
		\begin{tikzpicture} %
		\begin{axis}[smooth,
		xlabel={$training$}, %
		ylabel={$testing$}, %
		height=3cm,%
		width=6cm,%
		line width=1,%
		] %
		\addplot+[mark=square,fill=red!20!white] table[x={loss} , y={val_loss}] {\dnwsxoptinit};					
		\end{axis} %
		\end{tikzpicture}
	}
	\end{tabular}
	\caption {\textit{\trtscaption}}
\end{figure}

%%%%%%%%%%%%%%%%%%%%%%%%%%%%%%TRAINING VS TESTING ACCURACY AND LOSS FOR ALL BATCH SIZES%%%%%%%%%%%%%%%%%%%%%%%%%
%%%%%%%%%%%%%%%%%%%%%%%%%%%%%%%%%%%%%%%%%%%%%%%%%%%%%%%END%%%%%%%%%%%%%%%%%%%%%%%%%%%%%%%%%%%%%%%%%%%%%%%%%%%


\pgfplotstableclear{\dnwonoptinit}
\pgfplotstableclear{\dnwtwoptinit}
\pgfplotstableclear{\dnwfooptinit}
\pgfplotstableclear{\dnwsxoptinit}
\pgfplotstableclear{\dnwegoptinit}
\pgfplotstableclear{\dnwhnoptinit}

\relax


\subsubsection{CIFAR-10}
\textbf{Five Layer, opti=adagrad, batch size=2xc}

\pgfplotstableclear{\dnwonoptinit}
\pgfplotstableclear{\dnwtwoptinit}
\pgfplotstableclear{\dnwfooptinit}
\pgfplotstableclear{\dnwsxoptinit}
\pgfplotstableclear{\dnwegoptinit}
\pgfplotstableclear{\dnwhnoptinit}

\pgfplotstableread[col sep = comma]{ResultsNormal/cifar10_FiveLayer_20B_0E_adagrad_gl_n.csv}\dnwonoptinit
\pgfplotstableread[col sep = comma]{ResultsNormal/cifar10_FiveLayer_20B_0E_adagrad_gl_u.csv}\dnwtwoptinit
\pgfplotstableread[col sep = comma]{ResultsNormal/cifar10_FiveLayer_20B_0E_adagrad_he_u.csv}\dnwfooptinit
\pgfplotstableread[col sep = comma]{ResultsNormal/cifar10_FiveLayer_20B_0E_adagrad_norm.csv}\dnwsxoptinit
\pgfplotstableread[col sep = comma]{ResultsNormal/cifar10_FiveLayer_20B_0E_adagrad_unif.csv}\dnwegoptinit
\pgfplotstableread[col sep = comma]{ResultsNormal/cifar10_FiveLayer_20B_0E_adagrad_zero.csv}\dnwhnoptinit
\renewcommand{\initerr}{Error for initiaNITCIFAR10}
\renewcommand{\initacc}{Accuracy for initial epcohs}
\renewcommand{\lateerr}{Error for later epochs}
\renewcommand{\lateacc}{Accuracy for later epochs}
\renewcommand{\boxerr}{Error variation}
\renewcommand{\boxacc}{Accuracy variation}

\renewcommand{\initfigtitle}{Different batch results for starting 15 epochs }
\renewcommand{\initcaption}{to write}

\renewcommand{\latefigtitle}{Different batch results for later epochs}
\renewcommand{\latecaption}{to write}

\renewcommand{\boxerraccfigtitle}{accuracy and error plot for full training epochs}
\renewcommand{\boxerracccaption}{to write}

\renewcommand{\trtscaption}{to write}
\renewcommand	{\ymaxlim}{0.65}

\begin{figure}[H]
	\hspace{-1cm}
	\begin{tabular}{C{.5\textwidth}C{.5\textwidth}}%
		\subfigure [\initerr] {
			\begin{tikzpicture} %
			\begin{axis}[smooth,
			%xlabel={$epochs$}, %
			ylabel={$error$}, %
			height=4cm,%
			width=4cm,%
			legend style={font=\tiny,at={(1,1)},anchor=north east,fill=none, draw=none},%
			xmax=15,%
			yticklabel style={/pgf/number format/.cd,fixed,precision=5},%
			label style = {scale=0.8},%
			tick label style = {scale=0.8},%
			] %
			\addplot+[mark=pentagon*] table[x expr=\coordindex , y={val_loss}, col sep=comma] {\dnwonoptinit};
			\addlegendentry{1xc};%	
			\addplot+[mark=ball] table[x expr=\coordindex , y={val_loss}, col sep=comma] {\dnwtwoptinit};
			\addlegendentry{2xc};%	
			\addplot+[mark=10-pointed star] table[x expr=\coordindex , y={val_loss}, col sep=comma] {\dnwfooptinit};
			\addlegendentry{4xc};%							
			\addplot+[mark=triangle*] table[x expr=\coordindex , y={val_loss}, col sep=comma] {\dnwsxoptinit};
			\addlegendentry{6xc};%							
			\addplot+[mark=halfcircle*] table[x expr=\coordindex , y={val_loss}, col sep=comma] {\dnwegoptinit};
			\addlegendentry{8xc};%							
			\addplot+[mark=otimes*] table[x expr=\coordindex , y={val_loss}, col sep=comma] {\dnwhnoptinit};
			\addlegendentry{10xc};%
			\end{axis} %
			\end{tikzpicture}
		}&
			\subfigure [\initacc] {
			\begin{tikzpicture} %
			\begin{axis}[smooth,
			%xlabel={$epochs$}, %
			ylabel={$accuracy$}, %
			height=4cm,%
			width=4cm,%
			legend style={font=\tiny,at={(1,0)},anchor=south east,fill=none, draw=none},%
			xmax=15,%
			yticklabel style={/pgf/number format/.cd,fixed,precision=5},%
			label style = {scale=0.8},%
			tick label style = {scale=0.8},%
			] %
			\addplot+[mark=pentagon*] table[x expr=\coordindex , y={val_acc}, col sep=comma] {\dnwonoptinit};
			\addlegendentry{1xc};%	
			\addplot+[mark=ball] table[x expr=\coordindex , y={val_acc}, col sep=comma] {\dnwtwoptinit};
			\addlegendentry{2xc};%	
			\addplot+[mark=10-pointed star] table[x expr=\coordindex , y={val_acc}, col sep=comma] {\dnwfooptinit};
			\addlegendentry{4xc};%							
			\addplot+[mark=triangle*] table[x expr=\coordindex , y={val_acc}, col sep=comma] {\dnwsxoptinit};
			\addlegendentry{6xc};%							
			\addplot+[mark=halfcircle*] table[x expr=\coordindex , y={val_acc}, col sep=comma] {\dnwegoptinit};
			\addlegendentry{8xc};%							
			\addplot+[mark=otimes*] table[x expr=\coordindex , y={val_acc}, col sep=comma] {\dnwhnoptinit};
			\addlegendentry{10xc};%
			\end{axis} %
			\end{tikzpicture}
		}\\%
	\end{tabular}%
	\caption {\textit{\initfigtitle}}%
	\medskip
	\small
	\textit{\initcaption}
\end{figure}

\begin{figure}[H]
	\hspace{-1cm}
	\begin{tabular}{C{.5\textwidth}C{.5\textwidth}}%
		\subfigure [\boxerr] {
			\begin{tikzpicture} %
			\begin{axis}[%
			xlabel={\textit{batch size [in multiple of classes]}}, %
			ylabel={\textit{error}}, %
			boxplot/draw direction=y,%
			xtick={1,2,3,4,5,6},%
			xticklabels={1,2, 4, 6, 8, 10},%
			height=4cm,%
			width=6cm,%
			] %
			\addplot+ [boxplot,fill=cyan!10!white] table [x expr=1,y ={val_loss}] {\dnwonoptinit};							
			\addplot+ [boxplot,fill=green!20!white] table [x expr=2,y ={val_loss}] {\dnwtwoptinit};
			\addplot+ [boxplot,fill=orange!20!white] table [x expr=4,y ={val_loss}] {\dnwfooptinit};		
			\addplot+ [boxplot,fill=violet!10!white] table [x expr=6,y ={val_loss}] {\dnwsxoptinit};					
			\addplot+ [boxplot,fill=pink!10!white] table [x expr=8,y ={val_loss}]  {\dnwegoptinit};						
			\addplot+ [boxplot,fill=green!10!white] table [x expr=10,y ={val_loss}] {\dnwhnoptinit};
			\end{axis} %
			\end{tikzpicture}
		}&
		\subfigure [\boxacc] {
			\begin{tikzpicture} %
			\begin{axis}[%
			boxplot/draw direction=y,%
			xtick={1,2,3,4,5,6},%
			xticklabels={1,2, 4, 6, 8, 10},%
			xlabel={\textit{batch size [in multiple of classes]}}, %
			ylabel={\textit{accuracy}}, %
			height=4cm,%
			width=6cm,%
			] %
			\addplot+ [boxplot,fill=cyan!10!white] table [x expr=1,y ={val_acc}] {\dnwonoptinit};							
			\addplot+ [boxplot,fill=green!20!white] table [x expr=2,y ={val_acc}] {\dnwtwoptinit};
			\addplot+ [boxplot,fill=orange!20!white] table [x expr=4,y ={val_acc}] {\dnwfooptinit};		
			\addplot+ [boxplot,fill=violet!10!white] table [x expr=6,y ={val_acc}] {\dnwsxoptinit};					
			\addplot+ [boxplot,fill=pink!10!white] table [x expr=8,y ={val_acc}]  {\dnwegoptinit};						
			\addplot+ [boxplot,fill=green!10!white] table [x expr=10,y ={val_acc}] {\dnwhnoptinit};
			\end{axis}%
			\end{tikzpicture}
		}
	\end{tabular}
	\caption {\textit{\boxerraccfigtitle}}
	\medskip
	\small
	\textit{\boxerracccaption}
\end{figure}

\begin{figure}[H]
\hspace{-1cm}
\begin{tabular}{C{\textwidth}}%
	\subfigure [\lateerr] {
		\begin{tikzpicture} %
		\begin{axis}[smooth,
		xlabel={$epochs$}, %
		ylabel={$error$}, %
		height=3cm,%
		width=12cm,%
		legend style={font=\tiny,legend columns=-1, at={(1,1)},anchor=north east,fill=none, draw=none},%
		xmin=15,%
		yticklabel style={/pgf/number format/.cd,fixed,precision=5},%
		] %
		\addplot+[mark=pentagon*] table[x expr=\coordindex , y={val_loss}, col sep=comma] {\dnwonoptinit};
		\addlegendentry{1xc};%	
		\addplot+[mark=ball] table[x expr=\coordindex , y={val_loss}, col sep=comma] {\dnwtwoptinit};
		\addlegendentry{2xc};%	
		\addplot+[mark=10-pointed star] table[x expr=\coordindex , y={val_loss}, col sep=comma] {\dnwfooptinit};
		\addlegendentry{4xc};%							
		\addplot+[mark=triangle*] table[x expr=\coordindex , y={val_loss}, col sep=comma] {\dnwsxoptinit};
		\addlegendentry{6xc};%							
		\addplot+[mark=halfcircle*] table[x expr=\coordindex , y={val_loss}, col sep=comma] {\dnwegoptinit};
		\addlegendentry{8xc};%							
		\addplot+[mark=otimes*] table[x expr=\coordindex , y={val_loss}, col sep=comma] {\dnwhnoptinit};
		\addlegendentry{10xc};%
		\end{axis} %
		\end{tikzpicture}
	}\\
	\subfigure [\lateacc] {
		\begin{tikzpicture} %
		\begin{axis}[smooth,
		xlabel={$epochs$}, %
		ylabel={$accuracy$}, %
		height=3cm,%
		width=12cm,%
		legend style={font=\tiny,legend columns=-1, at={(1,0)},anchor=south east,fill=none, draw=none},%
		%ymin=\ymaxlim,%
		xmin=12,%
		yticklabel style={/pgf/number format/.cd,fixed,precision=5},%
		] %
		\addplot+[mark=pentagon*] table[x expr=\coordindex , y={val_acc}, col sep=comma] {\dnwonoptinit};
		\addlegendentry{1xc};%	
		\addplot+[mark=ball] table[x expr=\coordindex , y={val_acc}, col sep=comma] {\dnwtwoptinit};
		\addlegendentry{2xc};%	
		\addplot+[mark=10-pointed star] table[x expr=\coordindex , y={val_acc}, col sep=comma] {\dnwfooptinit};
		\addlegendentry{4xc};%							
		\addplot+[mark=triangle*] table[x expr=\coordindex , y={val_acc}, col sep=comma] {\dnwsxoptinit};
		\addlegendentry{6xc};%							
		\addplot+[mark=halfcircle*] table[x expr=\coordindex , y={val_acc}, col sep=comma] {\dnwegoptinit};
		\addlegendentry{8xc};%							
		\addplot+[mark=otimes*] table[x expr=\coordindex , y={val_acc}, col sep=comma] {\dnwhnoptinit};
		\addlegendentry{10xc};%
		\end{axis} %
		\end{tikzpicture}
	}\\
\end{tabular}
\caption {\textit{\latefigtitle}}
\medskip
\small
\textit{\latecaption}
\end{figure}

%

%%%%%%%%%%%%%%%%%%%%%%%%%%%%%%%%%%%%%%%%%%%%%%%%%%%%%%%START%%%%%%%%%%%%%%%%%%%%%%%%%%%%%%%%%%%%%%%%%%%%%%%%%%%
%%%%%%%%%%%%%%%%%%%%%%%%%%%%%%TRAINING VS TESTING ACCURACY AND LOSS FOR ALL BATCH SIZES%%%%%%%%%%%%%%%%%%%%%%%%%
\begin{figure}[H]
	\hspace{-1cm}
	\begin{tabular}{C{.50\textwidth}C{.50\textwidth}}%
		\subfigure [2x-Error] {
			\begin{tikzpicture} %
			\begin{axis}[smooth,
			label style = {scale=0.5},%
			tick label style = {scale=0.5},%
			xlabel={$training$}, %
			ylabel={$testing$}, %
			height=3cm,%
			width=6cm,%
			line width=1,%
			axis y line*=left,%
			legend style={legend columns=-1, at={(1,0)},anchor=south east,fill=red!10!white, draw=none},%
			] %
			\addplot+[blue!80!white,densely dotted,mark=1,fill=red!10!white] table[x={acc} , y={val_acc}] {\dnwtwoptinit};%
			\addlegendentry{accuracy};%						
			\end{axis} %
			\begin{axis}[smooth,
			label style = {scale=0.5},%
			tick label style = {scale=0.5},%
			%xlabel={$training-error$}, %
			%ylabel={$testing-error$}, %
			xticklabel shift={0.5cm},%
			axis line shift=3pt,%
			height=3cm,%
			width=6cm,%
			line width=1,%
			axis y line*=right,%
			legend style={legend columns=-1, at={(0,1)},anchor=north west,fill=blue!10!white, draw=none},%
			] %
			\addplot+[red!80!white,densely dotted,mark=1,fill=blue!10!white] table[x={loss} , y={val_loss}]{\dnwtwoptinit};%
			\addlegendentry{loss};%							
			\end{axis} %
			\end{tikzpicture}			
		}&
		\subfigure [2x-Accuracy] {
			\begin{tikzpicture} %
			\begin{axis}[smooth,
			label style = {scale=0.8},%
			xlabel={$training-error$}, %
			ylabel={$testing-error$}, %
			height=3cm,%
			width=6cm,%
			line width=1,%
			axis y line*=right,%
			legend style={legend columns=-1, at={(0,1)},anchor=north west,fill=blue!10!white, draw=none},%
			] %
			\addplot+[red!80!white,densely dotted,mark=1,fill=blue!10!white] table[x={loss} , y={val_loss}]{\dnwtwoptinit};%
			\addlegendentry{loss};%							
			\end{axis} %
			\end{tikzpicture}
		}\\%
	\subfigure [4x-Error] {
		\begin{tikzpicture} %
		\begin{axis}[smooth,
		xlabel={$training$}, %
		ylabel={$testing$}, %
		height=3cm,%
		width=6cm,%
	    line width=1,%
		] %
		\addplot+[blue!70!white,densely dotted,mark=square,fill=red!20!white] table[x={acc} , y={val_acc}] {\dnwfooptinit};				
		\end{axis} %
		\end{tikzpicture}%			
	}&
	\subfigure [4x-Accuracy] {
		\begin{tikzpicture} %
		\begin{axis}[smooth,
		xlabel={$training$}, %
		ylabel={$testing$}, %
		height=3cm,%
		width=6cm,%
		line width=1,%
		] %
		\addplot+[red!70!white,densely dotted,mark=square,fill=blue!20!white] table[x={loss} , y={val_loss}] {\dnwfooptinit};					
		\end{axis} %
		\end{tikzpicture}
	}\\%
	\subfigure [6x-Error] {
		\begin{tikzpicture} %
		\begin{axis}[smooth,
		xlabel={$training$}, %
		ylabel={$testing$}, %
		height=3cm,%
		width=6cm,%
		line width=1,%
		%ymin=0.992,%
		] %
		\addplot+[mark=square,fill=red!20!white] table[x={acc} , y={val_acc}] {\dnwsxoptinit};				
		\end{axis} %
		\end{tikzpicture}%			
	}&
	\subfigure [6x-Accuracy] {
		\begin{tikzpicture} %
		\begin{axis}[smooth,
		xlabel={$training$}, %
		ylabel={$testing$}, %
		height=3cm,%
		width=6cm,%
		line width=1,%
		] %
		\addplot+[mark=square,fill=red!20!white] table[x={loss} , y={val_loss}] {\dnwsxoptinit};					
		\end{axis} %
		\end{tikzpicture}
	}
	\end{tabular}
	\caption {\textit{\trtscaption}}
\end{figure}

%%%%%%%%%%%%%%%%%%%%%%%%%%%%%%TRAINING VS TESTING ACCURACY AND LOSS FOR ALL BATCH SIZES%%%%%%%%%%%%%%%%%%%%%%%%%
%%%%%%%%%%%%%%%%%%%%%%%%%%%%%%%%%%%%%%%%%%%%%%%%%%%%%%%END%%%%%%%%%%%%%%%%%%%%%%%%%%%%%%%%%%%%%%%%%%%%%%%%%%%


\pgfplotstableclear{\dnwonoptinit}
\pgfplotstableclear{\dnwtwoptinit}
\pgfplotstableclear{\dnwfooptinit}
\pgfplotstableclear{\dnwsxoptinit}
\pgfplotstableclear{\dnwegoptinit}
\pgfplotstableclear{\dnwhnoptinit}

\relax

\subsubsection{MNIST}
\textbf{Two Layer, opti=nadam, batch size=1xc}



